% Created 2020-04-14 Tue 13:06
% Intended LaTeX compiler: pdflatex
\documentclass[11pt]{article}
\usepackage[utf8]{inputenc}
\usepackage[T1]{fontenc}
\usepackage{graphicx}
\usepackage{grffile}
\usepackage{longtable}
\usepackage{wrapfig}
\usepackage{rotating}
\usepackage[normalem]{ulem}
\usepackage{amsmath}
\usepackage{textcomp}
\usepackage{amssymb}
\usepackage{capt-of}
\usepackage{hyperref}
\usepackage{amsmath}
\author{Berat Ertural 406055}
\date{16.04.20}
\title{Computational Physics Exercise 1}
\hypersetup{
 pdfauthor={Berat Ertural 406055},
 pdftitle={Computational Physics Exercise 1},
 pdfkeywords={},
 pdfsubject={},
 pdfcreator={Emacs 26.3 (Org mode 9.1.9)}, 
 pdflang={English}}
\begin{document}

\maketitle
\tableofcontents



\section{Derivative Formula}
\label{sec:org7b24678}

Let \(h\) be constant and let \(f(x-2h),f(x-h),f(x+h),f(x+2h)\) 
contain the Taylor expansion up to order 4 with the error term in \(O(h^5)\). 
We first construct the following;
\begin{equation}
f(x+h) + f(x-h) = 2f(x)+h^2 f^{''}(x)+ \frac{h^4}{12} f^{''''}(x) + O(h^6)
\end{equation}
Obtaining an error bound in \(O(h^6)\) since derivatives of uneven order cancel out. Similarily we get;
\begin{equation}
f(x+2h) + f(x-2h) = 2f(x)+4 h^2 f^{''}(x)+ \frac{16 h^4}{12} f^{''''}(x) +O(h^6)
\end{equation}

Multplying equation (1) by 16 and subtracting equation (2) we get the desired result;
\begin{multline}
-f(x+2h)+16f(x+h)+16f(x-h)-f(x-2h) = 30f(x)+ 12 h^2 f^{''}(x) + O(h^6) \\
\Rightarrow f^{''}(x) = \frac{1}{h^2}[-\frac{1}{12}f(x+2h)+\frac{4}{3}f(x+h)-\frac{5}{2}f(x)+\frac{4}{3}f(x-h)-\frac{1}{12}f(x-2h)] + O(h^4)
\end{multline}

Which is a \textbf{central 5-point formula} for the second derivative of \textbf{4-th order accuracy}.

\section{2. Simpson Rule}
\label{sec:org21d5b58}
\subsection{Simpson integration in Python}
\label{sec:org80be0c9}
The following function evaluates the integral of a given function \(f(x)\) on a finite interval 
\(x \in [a,b]\) using a set of \(n\) equidistant sample points.

\begin{verbatim}
import math

def simpson(f, a, b, n):
    """Approximate the Integral of f in the interval
    between a and b using n equidistant sample points using 
    Simpsons rule. The number of sample points can be even or odd."""
    # Enforce oddity
    n = max (n, 2)
    h = (b-a)/(n-1)
    odd_n = n -1 + n%2 # Enforce oddity
    S = f(a) + f(a+(odd_n-1)*h)
    # Integrate over alternating coefficients
    for i in range(1, n-1, 2):
	S += 4*f(a + i * h)
    for i in range(2, n-2, 2):
	S += 2*f(a + i * h)
    S *= h/3
    # Case n even
    if (n%2 == 0):
	S += (h/12)* (5*f(b) + 8*f(b-h) - f(b-2*h))
    return S

# Lets output a test value
simpson(math.sin, 0, math.pi/2, 8)
\end{verbatim}
\end{document}